\documentclass{beamer}
\usepackage[spanish]{babel}
\usepackage[utf8]{inputenc}
\usepackage{tikz}
\usepackage[T1]{fontenc} 
\usepackage{calligra} 
\usetikzlibrary{arrows,automata}



\input{Helpers.tex}
%\usetheme{Berkeley}
% \usetheme{JuanLesPins}
% \usetheme{bars}
% \usetheme{split}
  \usetheme[compress]{Ilmenau}
% \mode<presentation>
%\mode<handout>%{\beamertemplatesolidbackgroundcolor{black!5}}
% \mode<article>{\usepackage{fullpage}}
% \usepackage{pgfpages}
%\pgfpagelayout{resize}[a4paper,border shrink=5mm,landscape]

\include{epsfig}
\title{Supermercado D\'ia\%}

\author{Christian Russo y compañía}
\date{15 de Junio de 2015}

\begin{document}

\frame{\titlepage}

\frame{\tableofcontents}

\section{Dificultades encontradas}

\frame{
\frametitle{Dificultades encontradas} 
	\begin{itemize}
		\item No sabíamos cómo hacer la trazabilidad.
		\item No sabíamos que debíamos modelar con FSM y que con DA.
		\item La sincronizaci\'on de alarmas de falta de stock para los pedidos.
		\item Distinguir entre el rol de un local y el rol de un cliente (operaciones que pueden realizar, informaci\'on que se sabe de ambos para un pedido, etc.).
		\item C\'omo modelar los reportes de ventas.
		\item Asignar las responsabilidades de cada operaci\'on.
		\item Decidir cu\'ales agentes tendr\'an una cuenta para interactuar con el sistema.
	\end{itemize}
}

\section{Puntos fuertes del TP}

\frame{
\frametitle{Puntos fuertes del TP}
	\begin{itemize}
		\item La página web permite a los clientes realizar pedidos en forma simultánea.
		%\item Se puede configurar parámetros para el registro de ventas.
		\item Se puede configurar un umbral por producto para que se dispare una alarma que notifique al encargado de stock cuando el stock del mismo caiga por debajo de \'este.
		\item Los usuarios pueden ver el estado de su pedido en la p\'agina en todo momento.
		\item El sistema de penalizaci\'on de usuarios.
	\end{itemize}
}

 
\section{Puntos cuestionables del TP}
\frame{
\frametitle{Puntos cuestionables del TP}
	\begin{itemize}
	\item No adoptamos ninguna estrategia que aumente la probabilidad de que el usuario permanezca en su domicilio para recibir el pedido. Las alternativas que consideramos fueron: enviarle un SMS antes de que su pedido salga del depósito a modo de recordatorio y enviarle un SMS si su pedido se encontrara demorado con la nueva hora de llegada para que lo espere.
	\item No existe ninguna interacción entre los diferentes depósitos.
	\item El stock reservado por los pedidos puede superar al stock real del depósito. Esto podría traer dificultades para cumplir con la fecha de entrega pactada con el cliente.
	\end{itemize}
}

\section{Decisiones tomadas}
\frame{
\frametitle{Decisiones tomadas}
	\begin{itemize}
	\item Todos los locales tienen un dep\'osito asociado, éste maneja el stock de su correspondiente local.
	\item Cuando un cliente no est\'a en su domicilio, el sistema lo penaliza impidi\'endole realizar nuevos pedidos durante un per\'iodo de tiempo configurable.
	\item Los pedidos son asignados a un \'unico dep\'osito para procesarlos.
	\end{itemize}
}


\frame{
\frametitle{Fin}
	\begin{center}
	\huge{\textbf{?`Preguntas?}}
	\includegraphics[width=0.5\textwidth]{preguntas.png}
	\end{center}
}




\end{document}
