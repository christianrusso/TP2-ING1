\subsection{Trazabilidad}
Para mostrar la trazabilidad y la completitud de nuestros diagramas, armamos una tabla con los requerimientos y las presunciones del dominio aclarando para cada uno en que diagrama se ve reflejado. Luego en cada diagrama en particular se detalla cada requerimiento o presunción del dominio.
\begin{center}
\begin{table}[H]
\begin{tabular}{|l|l|l|}
\hline
\# & Requerimiento & Diagrama   \\ \hline
1  & Reponer los productos en las góndolas & \\ \hline
2  & Agregar depósitos al sistema & CU - MC \\ \hline
3  & Los usuarios se pueden autentificar. & CU - DA \\ \hline
4  & Que el cliente envié una copia del DNI mediante el sitio. & CU \\ \hline
5  & Cerrar un pedido desde la pagina web & CU - MC - DA \\ \hline
6  & Tomar productos del pedido => Notificar pedido armado. & CU - MC - DA \\ \hline
7  & Configurar umbrales por producto en el sistema. & CU - MC \\ \hline
8  & Incrementar stock cuando recibo reposición del proveedor. & CU - MC - FSM \\ \hline
9  & Coordinar fecha de entrega. & CU - MC - DA \\ \hline
10  & Se puede consultar los registros de las ventas.  & CU - MC \\ \hline
11  & Se puede configurar los parámetros del reporte. & CU - MC \\ \hline
12  & Mantener listado de productos comprados. &  MC \\ \hline
13  & Mantener información de los medios de pagos utilizados. & MC \\ \hline
14  & Los locales puedan registrarse. & CU - MC  \\ \hline
15  & Los locales pueden y ver y seleccionar los productos que desea recibir. & CU  - DA  \\ \hline
16  & Los locales pueden modificar los pedidos no cerrados. & CU - MC  \\ \hline
17  & Los clientes pueden y ver y seleccionar los productos que desean recibir. & CU  - FSM \\ \hline
18  & Los clientes pueden modificar los pedido no cerrados. & CU - MC  \\ \hline
19 & Un cliente pueda registrarse. & CU - MC  \\ \hline
20  & Un cliente pueda seleccionar la fecha de entrega. & CU - MC - DA  \\ \hline
21  & Un cliente pueda ingresar sus datos de contacto. & CU - MC  \\ \hline
22  & Reservar stock. & CU - MC - FSM \\ \hline
23  & Los productos en falta son filtrado del listado que ve el cliente. & CU  - FSM \\ \hline
24  & {\scriptsize La cantidad máxima permitida para pedir un producto es igual a la que hay en el deposito}. & CU  - FSM \\ \hline
25  &  {\scriptsize Lograr que la interfase funcione correctamente en computadoras de escritorio} &  \\ \hline
26  & Lograr que la interfase funcione correctamente en dispositivos móviles &  \\ \hline
27  & Entregar pedidos al local & DA \\ \hline
28  & Consultar pedidos armados & CU - MC  \\ \hline
29  & Seleccionar pedido entregado & CU - MC - DA  \\ \hline
30  & El cliente pago en contraentrega => se notifica pago a contraentrega. & CU - MC - DA  \\ \hline
31  & El cliente pago desde la pagina => se notifica el pago online. & CU - MC - DA \\ \hline
32  & Coordinar fecha de entrega con proveedor. & CU - DA  \\ \hline
33  & Proveedor envía producto & FSM \\ \hline
34  &  {\scriptsize Llevar el conteo en base a los pedidos realizados de cuantos productos quedan.} & CU - MC  - FSM \\ \hline
35  &  {\scriptsize Se disparan alarmas cuando la cantidad de productos cae debajo del umbral configurado}  & FSM \\ \hline
36 & Contactar al proveedor. & FSM \\ \hline
\end{tabular}
\end{table}
\end{center}


\begin{center}
\begin{table}[H]
\begin{tabular}{|l|l|l|}
37  &  {\scriptsize Poder definir si un cliente es malo en función de un porcentaje de pedidos no entregados.} &  \\ \hline
38  & Mantener historial de pedidos entregados y no entregados de un cliente.  & MC \\ \hline
39  & Incrementar el precio del próximo pedido. &  \\ \hline
40  & Rechazar pedido durante un tiempo & MC - FSM \\ \hline
41  & Pedido llega a la casa del cliente & MC - FSM \\ \hline
42  & Cliente pague el pedido & CU - MC  \\ \hline
43  & Ofrecer pago con PayPal & CU - MC - DA \\ \hline
44  & Ofrecer pago con crédito & CU - MC - DA \\ \hline
45  & Ofrecer pago con débito & CU - MC - DA  \\ \hline
46  & Ofrecer pago con efectivo.  & CU - MC - DA  \\ \hline
\end{tabular}
\end{table}
\end{center}

\textit{CU}: Diagrama de casos de uso.

\textit{MC}: Diagrama de clases - Modelo conceptual.

\textit{FSM}: Maquina de estados.

\textit{DA}: Diagrama de actividad.

\newpage
